\documentclass[12pt]{article}
\usepackage{bussproofs}
\usepackage{comment}
\usepackage{cmll}
\usepackage{amssymb}
\usepackage{amsthm}
\usepackage{latexsym}
\newtheorem{theorem}{Theorem}[section]
\newtheorem{lemma}[theorem]{Lemma}
\newtheorem{property}[theorem]{Property}
\newtheorem{corolary}[theorem]{Corolary}
\theoremstyle{definition}
\newtheorem{define}[theorem]{Definition}
\newcommand\nb[1]{{#1}^{\bot}}
\title{Linear-Logic Rule}
\author{Arthur MILCHIOR}
% This is the "centered" symbol
\def\fCenter{{\mbox{\Large$\rightarrow$}}}

% Optional to turn on the short abbreviations
\EnableBpAbbreviations

% \alwaysRootAtTop  % makes proofs upside down
% \alwaysRootAtBottom % -- this is the default setting

\begin{document}
\section{Linear logic}
\subsection{Rule of linear sequent calculus}

\subsubsection{Additive}
\begin{minipage}[l]{.50\linewidth}
\begin{prooftree}
\AxiomC{$\vdash\Gamma,A$}
\AxiomC{$\vdash\Gamma,B$}
\RightLabel{$\with$}
\BinaryInfC{$\vdash\Gamma,A\with B$}
\end{prooftree}\end{minipage}
\begin{minipage}[l] {.50\linewidth}
\begin{prooftree}
\AxiomC{$\vdash\Gamma,A_i$}
\RightLabel{$\oplus_i$}
\UnaryInfC{$\vdash\Gamma,A_1\oplus A_2$}
\end{prooftree}\end{minipage} 

\begin{minipage}[l]{.50\linewidth} 
\begin{prooftree}
\AxiomC{$\vdash\Gamma,A$}
\RightLabel{$\forall$, $x$ free in $\Gamma$}
\UnaryInfC{$\Gamma,\forall x: A$}
\end{prooftree}\end{minipage}
\begin{minipage}[l] {.50\linewidth}
\begin{prooftree}
\AxiomC{$\vdash\Gamma,A$}
\RightLabel{$\exists$}
\UnaryInfC{$\Gamma,\exists x: A$}
\end{prooftree}\end{minipage}

\begin{minipage}[l] {.50\linewidth}
\begin{prooftree}
\AxiomC{}
\RightLabel{$\top$}
\UnaryInfC{$\vdash\Gamma,\top$}
\end{prooftree}
\end{minipage}

\begin{minipage}[l]{.50\linewidth}
There is no for 0.
\end{minipage}

\subsubsection{Multiplicative}

\begin{minipage}[l]{.50\linewidth}
\begin{prooftree}
\AxiomC{$\vdash\Gamma,A$}
\AxiomC{$\vdash\Delta,B$}
\RightLabel{$\otimes$}
\BinaryInfC{$\vdash\Gamma,\Delta,A\otimes B$}
\end{prooftree}\end{minipage}
\begin{minipage}[l] {.50\linewidth}
\begin{prooftree}
\AxiomC{$\vdash\Gamma,A_1,A_2$}
\RightLabel{$\parr$}
\UnaryInfC{$\vdash\Gamma,A_1\parr A_2$}
\end{prooftree}\end{minipage} 

\begin{minipage}[l] {.50\linewidth}
\begin{prooftree}
\AxiomC{}
\RightLabel{$1$}
\UnaryInfC{$\vdash1$}
\end{prooftree}\end{minipage}
\begin{minipage}[l] {.50\linewidth}
\begin{prooftree}
\AxiomC{$\vdash\Gamma$}
\RightLabel{$\bot$}
\UnaryInfC{$\vdash\Gamma,\bot$}
\end{prooftree}\end{minipage}

There is usually no multiplicative quantifiers in linear logic.

\subsubsection{Exponentials}
\begin{minipage}[l] {.50\linewidth}
\begin{prooftree}
\AxiomC{$\vdash\Gamma,?A,?A$}
\RightLabel{contraction}
\UnaryInfC{$\vdash\Gamma,?A$}
\end{prooftree}\end{minipage} 
\begin{minipage}[l] {.50\linewidth}
\begin{prooftree}
\AxiomC{$\vdash\Gamma$}
\RightLabel{Weakening}
\UnaryInfC{$\vdash\Gamma,?A$}
\end{prooftree}\end{minipage}
\begin{minipage}[l] {.50\linewidth}
\begin{prooftree}
\AxiomC{$\vdash\Gamma,A$}
\RightLabel{Derilection}
\UnaryInfC{$\vdash\Gamma,?A$}
\end{prooftree}\end{minipage} 
\begin{minipage}[l] {.50\linewidth}
\begin{prooftree}
\AxiomC{$\vdash?\Gamma,A$}
\RightLabel{Promotion}
\UnaryInfC{$\vdash?\Gamma,!A$}
\end{prooftree}\end{minipage}

\subsubsection{Identity}

\begin{minipage}[l]{.50\linewidth}
\begin{prooftree}
\AxiomC{}
\RightLabel{axiom}
\UnaryInfC{$\vdash A,\nb{A}$}
\end{prooftree}\end{minipage}
\begin{minipage}[l] {.50\linewidth}
\begin{prooftree}
\AxiomC{$\vdash\Gamma,A$}
\AxiomC{$\vdash \nb{A},\Delta$}
\RightLabel{cut}
\BinaryInfC{$\vdash \Gamma,×\Delta$}
\end{prooftree}\end{minipage}

\begin{minipage}[l]{.50\linewidth}
\begin{prooftree}
\AxiomC{$\nb{A}\vdash\Gamma$}
\RightLabel{$\nb{R}$}
\UnaryInfC{$\vdash A,\Gamma$}
\end{prooftree}\end{minipage}
\begin{minipage}[l]{.50\linewidth}
\begin{prooftree}
\AxiomC{$\vdash\nb{A},\Gamma$}
\RightLabel{$\nb{L}$}
\UnaryInfC{$A\vdash\Gamma$}
\end{prooftree}\end{minipage}
\end{document}
