
\documentclass[12pt]{article}
\usepackage{bussproofs}
\usepackage{comment}
\usepackage{cmll}
\usepackage{amssymb}
\usepackage{amsthm}
\usepackage{latexsym}
\usepackage{amsmath}
\usepackage{centernot}
\usepackage{color}
\newtheorem{theorem}{Theorem}[section]
\newtheorem{lemma}[theorem]{Lemma}
\newtheorem{property}[theorem]{Property}
\newtheorem{corolary}[theorem]{Corolary}
\theoremstyle{definition}
\newtheorem{define}[theorem]{Definition}
\newcommand\nb[1]{{#1}^{\bot}}
\title{Linear-Logic Rule}
\author{Arthur MILCHIOR}
% This is the "centered" symbol
\def\fCenter{{\mbox{\Large$\rightarrow$}}}

% Optional to turn on the short abbreviations
\EnableBpAbbreviations

% \alwaysRootAtTop  % makes proofs upside down
% \alwaysRootAtBottom % -- this is the default setting

\begin{document}
\section{Vacuously True}

\begin{tabular}{ l | c | r }
    P & Q & $P \implies Q$ \\
    \hline			
    T & T & T \\ \hline
    T & F & F \\ \hline
    F & T & \color{red}T \\ \hline
    F & F & \color{red}T \\
    \hline  
  \end{tabular}

  Because of Biconditional being the equality in logic vacuous statements are true. False must be equivalent to false, create both conditionals being vacuously true to accomplish logical equivalence.

\end{document}